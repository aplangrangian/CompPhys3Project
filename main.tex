\documentclass{homework}

\title{Quantum Mechanics II Computational Project}
\author{Alexander Lange, Sara Ratliff, Roman Kosarzycki}

\begin{document}

\maketitle

In order to perform the Gaussian integration, we used the following relation:

\begin{equation}
\label{1}
\int^{\infty}_0 dq \ f(q) = \sum^N_{i=1} \tilde{\omega}_if_i + R_N[f,q] \ , \ f_i=f(q_i) \ , \ \tilde{\omega}_i = \omega_i \left[ \frac{dq}{dx} \right]_{x=x_i}
\end{equation}

The term $R_N$ represents the remainder, but we will choose a sufficiently large $N$ such that the remainder is negligible. For our purposes, we will use the following definition of $q$:

\begin{equation}
\label{2}
q(x)= q_0 \frac{1+x}{1-x}
\end{equation}

Eq.~(\ref{2}) allows $q$ to vary from 0 to $\infty$ when $x$ varies from -1 to 1. 

\end{document}
