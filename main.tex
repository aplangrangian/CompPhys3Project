\documentclass{homework}

\title{Quantum Mechanics II Computational Project}
\author{Alexander Lange, Sara Ratliff, Roman Kosarzycki}

\begin{document}

\maketitle

To perform Gauss-Jordan elimination with pivoting, we will consider an inhomogeneous system of $N$ linear equations, which can be represented in the following matrix form:

\begin{equation}
\label{1}
\begin{pmatrix}
a_{11} & a_{12} & \cdots & a_{1N} \\ a_{21} & a_{22} & \cdots & a_{2N} \\ \vdots & \vdots & \ddots & \vdots \\ a_{N1} & a_{N2} & \cdots & a_{NN}
\end{pmatrix}
\begin{pmatrix}
x_1 \\ x_2 \\ \vdots \\ x_N
\end{pmatrix} = 
\begin{pmatrix}
b_1 \\ b_2 \\ \vdots \\ b_N
\end{pmatrix}
\end{equation}

The Gauss-Jordan algorithm transforms the $N \times N$ matrix $\mathbf{A}$ into a triangular matrix:

\begin{equation}
\label{2}
\begin{pmatrix}
a'_{11} & a'_{12} & \cdots & a'_{1N} \\ 0 & a'_{22} & \cdots & a'_{2N} \\ \vdots & \vdots & \ddots & \vdots \\ 0 & 0 & \cdots & a'_{NN}
\end{pmatrix}
\begin{pmatrix}
x_1 \\ x_2 \\ \vdots \\ x_N
\end{pmatrix} = 
\begin{pmatrix}
b'_1 \\ b'_2 \\ \vdots \\ b'_N
\end{pmatrix}
\end{equation}

First, we created the augment matrix $\mathbf{\tilde{A}}$ by appending the vector $\mathbf{b}$ to form a $N \times (N+1)$ matrix:

\begin{equation}
\label{3}
\mathbf{\tilde{A}} =
\begin{pmatrix}
a_{11} & a_{12} & \cdots & a_{1N} & b_1 \\ a_{21} & a_{22} & \cdots & a_{2N} & b_2 \\ \vdots & \vdots & \ddots & \vdots  & \vdots \\ a_{N1} & a_{N2} & \cdots & a_{NN} & b_N
\end{pmatrix}
\end{equation}

Given this augment matrix, we then go row by row, first pivoting and then transforming each variable $a_{ji}$. Pivoting is the process of finding the largest value in the first column $a_{j1}$ (including the last column) and moving that whole row to the $j^{th}$ row. Then each of individual components are transformed according to the following:

\begin{equation}
\label{4}
a_{ij} \to a'_{ij} = a_{ij} - \frac{a_{ik}}{a'_{kk}} a'_{kj} \ ; \ k=1,2, \cdots , N \ ; \ i = k+1 , k+2 , \cdots , N \ ; \ j = 1,2, \cdots , N+1
\end{equation}

In order to perform the Gaussian integration, we used the following relation:

\begin{equation}
\label{a}
\int^{\infty}_0 dq \ f(q) = \sum^N_{i=1} \tilde{\omega}_if_i + R_N[f,q] \ , \ f_i=f(q_i) \ , \ \tilde{\omega}_i = \omega_i \left[ \frac{dq}{dx} \right]_{x=x_i}
\end{equation}

Our code goes through this process in the following way:

The term $R_N$ represents the remainder, but we will choose a sufficiently large $N$ such that the remainder is negligible. For our purposes, we will use the following definition of $q$:

\begin{equation}
\label{b}
q(x)= q_0 \frac{1+x}{1-x}
\end{equation}

Eq.~(\ref{b}) allows $q$ to vary from 0 to $\infty$ when $x$ varies from -1 to 1. 

\end{document}
